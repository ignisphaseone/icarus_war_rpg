\documentclass{article}
\usepackage[top=4.5cm, bottom=4.5cm, left=2cm, right=2cm]{geometry}
\usepackage{fancyhdr} % This should be set AFTER setting up the page geometry
\usepackage{multicol}
\setlength{\columnsep}{13mm}
\usepackage{lipsum}
\pagestyle{fancy} % options: empty , plain , fancy
\renewcommand{\headrulewidth}{0pt} % customise the layout...
\lhead{Eric Fong}\chead{Dragonstorm}\rhead{\thepage}
\lfoot{}\cfoot{}\rfoot{}

\title{\bfseries Icarus War}

\author {Eric Fong}
\begin{document}
\maketitle
\copyright{This work is licensed under the Creative Commons
Attribution-NonCommercial-ShareAlike 3.0 United States License. To view a copy
of this license, visit http://creativecommons.org/licenses/by-nc-sa/3.0/us/ or
send a letter to Creative Commons, 171 Second Street, Suite 300, San Francisco,
California, 94105, USA.}
\thispagestyle{empty}
\clearpage
\begin{multicols}{2}
\tableofcontents
\thispagestyle{empty}
\clearpage
\pagestyle{fancy}

\setcounter{page}{1}
\section{Introduction}
If you’re reading this, you must be a select few people who understand what
role-playing is! Congratulations, you’re delving into a new system. Feel free
to use as much or as little as you want. Just looking for a world? The world
background and atmosphere will make a great campaign setting for you. Looking
for a system for character progression and combat? The system is designed for
modern combat, but can easily be shoe-horned into any other setting. I hope you
enjoy it!
\subsection{The Icarus War World}
\subsubsection{Description}
\subsubsection{Society}
\subsubsection{The Runners}
\subsubsection{Government}
\subsubsection{The Objective}
\subsection{Influences}
One of the most influential aspects of this RPG is a video game released in
2008 called Mirror’s Edge, developed by DICE and published by Electronic Arts.
It was a first person shooter, but took the emphasis off of the gunplay and
instead put the emphasis on parkour and acrobatics. The city in the game is
unnamed; the city of New Tokyo is based heavily off of the city in Mirror’s
Edge. Faith Connors, one of the sample characters, is the main character from
Mirror’s Edge.

Another influence in style and setting is Shadowrun. One common
theme that Icarus War shares with Shadowrun is people versus a bigger force. In
both games, it’s about the small players exerting their power and bringing down
other aspects. One major difference is that Shadowrun puts emphasis on guns and
firepower being the primary forces of change. Instead, Icarus War has the
players change the world indirectly, through sabotage, deceit, and winning on
small fronts.

Another role playing game that is an influence is Savage Worlds.
It works best with this kind of combat system, which involves fast paced combat
and traversal of territory. Savage Worlds is an open system that lets a GM come
up with a lot of setting and story; much of the Icarus War setting and story
was developed in a Savage Worlds campaign.

\section{Getting Started}
\subsection{History of the World}
The world hasn’t changed much, with the notable exception of peace between
countries. When one nation finally created the perfect government, with the
right balance of censorship, security, and freedom, other governments began to
adopt it. But to some, any censorship and security is a crime worthy of
revolution.

It would have been impossible for the censors and security to become the world
standard if it weren’t for Pirandello Kreuger. A private security company
founded in Italy, they turned private security into an export. Ruthlessly
efficient, with excellent businessmen helming the company, PK became the name
in private security. They finally bought out other security firms, making sure
they operated efficiently. Soon, PK became the main private security force
worldwide.

When Italy signed a government contract that turned PK into the government
security force, people were upset. The separation between government protecting
the people, and PK holding a lot of confidential information, was considered
the one reason PK worked as a security force. With that separation gone, the
government and PK combined to be what most people considered a totalitarian
government. It was the last straw, and the people of Italy rioted right in
front of the Vatican.

Other governments did not learn their lessons either, or thought they had to
prepare for the riots themselves. Most other governments signed similar
contracts with the PK, and suffered their own riots in doing so. Worldwide, the
incidents are known as the November Riots, since they happened globally, in a
span of two weeks. Some governments handled the November Riots better than
others, while some became truly totalitarian governments in that time. The
November Riots laid the stage for the underground resistance, known as Runners.

For the first time, a global revolution crossed boundaries. The government,
while not global, was a common enemy between nations. Everyone that fought back
felt that they were doing their part for freedom. In the Middle East, people
set aside their differences and united against a common enemy. In China, the
government broke the back of the people, forcing them to choose who they would
align themselves with. In New Tokyo, the government tried their best to keep
their citizens as free as possible; it didn’t work, and their runners are some
of the most invasive of the movement.

\subsubsection{Privatization Era}
\subsubsection{The November Riots}
With security clearly private, controlling enough government assets to force
alteration of laws, the people had finally realized what happened under their
noses. In just a few years, the “government” had succumbed to corporate greed
and power, implementing harsh surveillance on peoples’ lives. People began to
protest peacefully, demanding some changes. When the private government
corporations responded with anti-riot water cannons, arrests, and violence, the
people turned their demonstration into a full-fledged riot.
\subsubsection{Rise of the Runners}
\subsubsection{The Icarus War}
With the existence of the runners across the world, people began to adjust to
their presence.

Robert Hope was a politician in New Tokyo, a prime mayoral candidate. He was
against the privatization of the government security forces, and commonly spoke
out against the censorship that the government had implemented. It was clear
that, nearing the election, Robert Hope was the popular candidate.

The Icarus War started when Robert Hope was murdered in his office. The media
originally blamed it on Kate Connors, the first CPF officer that arrived on the
scene after it was reported. What they didn’t know is that Faith Connors, one
of the best runners in New Tokyo, was Kate’s sister. Faith managed to find out
who was really responsible for the assassination of Robert Hope; Pirandello
Kreuger, the massive private security company that had a bid to take over the
government’s security. By blaming the CPF for Robert Hope’s murder, they
believed that they could decrease their bid for taking over the government’s
security operations. Instead, they brought down the wrath of the Runners.

PK had been developing an anti-runner project; because of the situation in New
Tokyo, they were forced to deploy it early. Project Icarus was a special
training regiment, to train police officers in pursuing and incapacitating
runners. Faith managed to kill two of the runners who had betrayed their skills
and talents to PK, putting a large damper on the completion of Project Icarus.
This conflict was the opening of the Icarus War.
\subsubsection{The World Today}
\subsection{Character Dossier}
If you just want to get an idea of how the system works, pick a sample character
from the list below, and look at the stats! You might not understand
everything, but it should be easy enough to get the general idea, and find a
sample character you like.
\subsubsection{Faith Connors}
Leader, Acrobatic, New Tokyo
\subsubsection{Martin Wells}
Leader, Acrobatic, New Tokyo, Pacifism

City Traversal: Once per encounter, Martin can move twice his distance away from the nearest enemies, as though he were “Panicked.” He is not Panicked.

Spirit of the City: Once per game, each character within double Martin’s move range may move twice their move range in any direction.
\subsubsection{Kate Connors}
Trainee, Shooting, Police Officer, Charismatic
\subsubsection{Erin Low}
Leader, Acrobatic, New Tokyo, Coordination

Team Coordination: Once per encounter, after all initiative cards have been revealed, Erin can swap any two of them.

Preparation: Once per game, all player characters start the round on hold.
\subsubsection{Scott Ralf}
Politician, Explosives, Criminal

Scott was a politician in the wrong place at the wrong time. The politician who hired him, Robert Hope, was assassinated. Although a police officer had already been implicated in his actual assassination, the government used the opportunity to arrest many people within Robert Hope’s organization. Scott Ralf was one of those people. As Robert Hope’s head of personnel, he took the brunt of a lot of accusations,  

Criminal Connections: Once per encounter, Scott can use his criminal connections to have supplies dropped at a specific location.

Armor Wrecker: Once per game, Scott can move twice his speed directly towards a vehicle, and attack it for double damage. If the attack does not hit, Scott can use Armor Wrecker again.
\subsubsection{Charlie Sanders}
Technician, Communications, Electronics

Electronic Sabotage: Once per encounter, Charlie can declare an electrical component to sabotage in advance. If Charlie does not declare sabotage in advance, she may use Electronic Sabotage as she encounters a device in game.

Communication Overmastery: Once per game, Charlie can block all enemy communications in the area, while tripling the range of friendly communications.
\subsubsection{Douglas Falcon}
Driver, Pilot, Speed, Dodge

Falcon Kick: Once per encounter, Douglas can move twice his speed towards an opponent and attack him for double damage. Douglas must move at least his speed in a straight line towards his opponent in order to use this ability. If Douglas misses, he may use Falcon Kick again.

Emergency Hotwire: Once per game, Douglas attempts to steal a vehicle on the street.
\subsubsection{Zach Thomas}
Sabotage, Thievery
\subsubsection{Sarah Walker}
Stealth, Disguise, Contacts
\subsubsection{Rodriguez Mallas}
Sniper, Stealth
\subsubsection{Koya Estani}
Acrobat, Signature Weapon (Knives), Running

Stealth Attack: Once per encounter, Koya can attack a target with +4 to hit and damage. If this attack does not hit, Koya may Stealth Attack again, but not on the same target.

Acrobatic Assault: Once per game, Koya can move twice. She can also throw knives at up to three targets within her attack range at any point during her movement. If performed during an acrobatic maneuver, the attacks gain the benefit of Stealth Attack.
\subsubsection{Miles Jackson}
Climber, Signature Weapon (Marbles), Veteran

Bag o\' Tricks: Once per encounter, Miles can throw down a bunch of marbles,
over a very large area, turning it into difficult terrain.

Buckshot Sling: Once per game, Miles can use a sling to throw a large number of marbles at his opponents. He attacks all people in a cone area in front of him.
\subsubsection{Ashley Mistelle}
Disguise, Contacts, Dancer, Acrobat, Pacifism
\subsubsection{Stella Tam}
Politics, Social, Communication, Messenger

Region: New Tokyo

Training: Runner Training

[Str d6], [Agi d8], [Int d6], [Spi d4], [Ins d8] (Stats without training/region bonuses.)

Investigation Department: Once per encounter, Stella can call her friends from the police department to aid in a runner investigation. It is the GM’s discretion on to what information Stella receives from her friends.

Police Scramble: Once per game, Stella can call in a favor in the police department, causing all police forces in the area to scramble in a high alert for a number of rounds. This can be used to call forces away from an area, or to have them assist in some other ways.

\subsubsection{Audrey Parker}
FBI, investigator

\section{Character Creation}
\subsection{Region}

\subsubsection{New Tokyo (Japan)}
When visiting New Tokyo, people normally comment on the sprawling cityscape, the various skyscrapers, and the clean white city. Anybody that follows outside news sources knows that New Tokyo is considered “runner capital,” the place considered the birthplace of the runners. The November Riots started in New Tokyo, along with the real rise of Pirandello Kreuger and private security. A select few people fell off the fringes of society, left homeless and helpless. A select few of those decided to do what little they could to live above the government censorship and bring down the man.

Runner Training:

Technical Training:

Stealth Training:
\subsubsection{Beijing (China)}
Runners in China have one main influence on their training; a truly oppressive government that can maintain control over the entire population and media infrastructure. The November Riots were mainly an excuse to clamp down on the population for good. One of the unique aspects of China was that Pirandello Kreuger actually hired a good portion of the government’s already existing workforce, meaning most government officials have some combination of communist and PK training.

The runners have been forced to evolve into a disconnected yet supportive group. With most avenues of communications heavily monitored and censored by the government, runners have resorted to a very specific system of rotating drop-sites. When the city runs a “sweep,” it makes a lot of the drop-sites invalid or polluted with PK traps and bogus information. Chinese runners put a lot of emphasis on solitary training and survival; aborting a mission is never bad, as long as it protects the safety of everyone involved. To them, it is opposite of the enduring Chinese philosophy, where every person is expendable because there are so many.

All Chinese runners get the following bonus attributes: +d2 Spirit, +d2 Instinct, 2 Stealth, 1 Technical, 2 Running. Chinese runners can be trained for certain tasks, and also gain the bonuses below.

Commando Training:

Sabotage Training:

Assassin Training:
\subsubsection{Florence (Europe)}
Surveillance Training:

Social Training:

Security Training:
\subsubsection{Los Angeles (USA)}
American runners have always had a very different lifestyle from other runners. Where runners in the rest of the world focus on athletics, stealth, and technical aspects of running, American running is much more about gunplay and social contacts, as it arose from the gang culture that developed in metropolitan areas. Gang members, already on the fringes of society, took up running to escape the streets and fight back against others, although sometimes gang warfare crosses runner lines, with runners shooting each other over gang territory.

The United States has always favored free speech over everything else; the world was surprised when they were the first government to privatize their security fully. The citizens still have legislation today trying to dethrone private security, but most of the opposition mysteriously disappears. With policemen out of a job, some of them became runners, those who truly believed in doing their jobs and did not agree with what the government was doing now. Most criminals already had the basic “training” required to be a runner, so sometimes the conflicts restricted to very specific parts of the metropolitan hellhole spread to the runner’s rooftops and back alleys.

Criminal Training:

Social Training:

Police Training:
\subsubsection{Moscow (Russia)}
Demolitions Training:

Science Training:

Infiltration Training:
\subsubsection{Sydney (Islander)}
Sniper Training:

Survival Training:

Political Training:
\subsubsection{Ivory Coast (Africa)}
Athletic Training:

Guerilla Training:

Spiritual Training:
\subsubsection{Other Regions}
\subsection{Character Stats}
\subsubsection{Strength}
\subsubsection{Agility}
\subsubsection{Intelligence}
\subsubsection{Spirit}
\subsubsection{Instinct}
\subsection{Experience}
Experience is accumulated in and out of game; more so in Icarus War, experience
is abstracted into units called marks. You get marks in certain categories, and
you can get these marks during a game session (if your character does a lot
during a session) or outside as session (if your character spends their off
time training).
\subsubsection{Running}
\subsubsection{Combat}
\subsubsection{Social}
\subsubsection{Technical}
\subsubsection{Stealth}
\subsection{RPG Character Creation}
\begin{enumerate}
  \item Pick where your character is from.
  \item Assign your character's stats.
  \item Apply your character’s origin and training package modifiers to your
  Stats, Traits, and Maneuvers.
  \item Look at your origin and training package to see which skills you
  start with.
  \item Purchase additional Stats, Traits, Maneuvers, and Skills with your
  training marks.
  \item Come up with two unique Character Abilities; one you use per encounter,
  and one you use per game. Work with the GM to get approval and not make
  anything too broken.
\end{enumerate}
\section{Skills}
\subsection{Running (Agility)}
\subsection{Combat (Strength)}
\subsection{Social (Spirit)}
\subsection{Technical (Intelligence)}
\subsection{Stealth (Instinct)}
\end{multicols}
\section{Conflict and Combat}
\subsection{Basics}
Checks: Pass means you do it, fail can mean two things: either your character
still succeeds but takes some sort of lingering penalty, or your character opts
not to perform the action at all because they know that they can’t do it.
\subsection{Momentum}

Momentum is used for allowing your character to perform more dramatic actions.

Characters often start with two momentum per encounter, although certain situations and character traits mean that some characters can start with more momentum.

You can only ever spend one momentum before a roll is attempted. You may spend as much momentum as you want after rolls.

You can spend a momentum before a roll to grant a +3 circumstance bonus on your roll, and negate the effects of a botch if you roll one.

You can spend a momentum before a roll to break a large skill die into a number of smaller ones for single actions. (You may not break your trait die.) The large die must be a d8, d10, or d12. They break into 2d4, d4+d6, and 2d6/3d4 respectively.

You can spend a momentum before a roll to break your actions into compound actions. Follow the rules for breaking dice above, and use the multiple dice you get from breaking for your separate actions. (Pair a trait die with one of the smaller die, for every part of the compound action you wish to take.)

You can spend a momentum after a roll is made to reroll the dice.

You can spend a momentum after a roll is made to add a +1 bonus to the roll.

Some abilities cost momentum to use, and sometimes have a minimum momentum requirement before you can use them.

You gain a momentum if a die you roll shows the highest value possible.

You gain one momentum for each die showing the max value.

Players can give you a momentum as an action or part of it. You can repeatedly perform this action until you run out of momentum.

You can spend your momentum to help a player you are assisting by letting them reroll their dice.

When you roll the highest value on a die, you keep the maximum value and get momentum, which is like a chip.

The Icarus War world is about movement and fluidity. When you start moving, it’s much easier to keep going faster than to slow down. Once a runner is on the move, it only takes a little effort to speed up, and it takes almost all of their effort to stop. Newton had it right; an object in motion tends to stay in motion.

He even had the right word for it; momentum. In this game, every runner understands the concept of momentum and uses it to survive. Momentum is used by players to adjust their rolls so that they can perform feats normally beyond their reach. Jumping between two buildings normally too far to reach, sprinting full speed over a thin construction beam, grabbing an enemy by the arm and flipping them onto the ground with a twist of your hand, or sliding under a quickly closing door are all examples of out-of-reach tasks that you can accomplish by using momentum. Momentum can even help in non-action situations, showing that you are slowly gaining the advantage in an interrogation or have slowly learned how to circumvent security on a stealth mission.

A player begins a scene with momentum. Often, they start with two momentum, but certain training or circumstances mean that characters can start off with more or less momentum.
\subsection{Damage}
There are three types of damage: glancing hits, wounding hits, and critical hits.

Glancing hits represent scrapes, bruises, and damaged mental health. Skidding on the ground, and bullets whizzing by your head are examples of glancing hits.

Wounding hits represent pretty major injuries, but nothing your character can’t handle. Getting thrown onto the ground by an enemy, getting shot in the arm, or spraining your ankle jumping are examples of wounding hits.

Critical hits represent crippling or life-threatening injuries. Getting shot in the leg or the gut, falling off a building, or being beaten unconscious are examples of critical hits.

If a character has taken the maximum number of hits of any given level, then hits of that level do the next level damage of severity. This is called wrap-around damage.

Rule of severity: if a character takes a wounding hit, they also take a glancing hit. If a character takes a critical hit, they also take a wounding hit and a glancing hit. Wrap around damage does not overlap with the rule of severity; rule of severity takes precedence and does not trigger wrap-around damage.

In this line of work, everyone gets hurt. Sometimes it’s something as mundane as landing incorrectly after a jump, other times, it’s getting shot at when running away from CPF forces. Of course, the risks are higher depending on which situation you find yourself stuck in. Tripping and twisting your ankle may prevent you from playing a critical part in a job, but you’ll live. Getting shot in the leg means that the CPF might catch you, which some people say is worse than death. Damage is broken into three tiers; glancing hits, wounding hits, and critical hits.
\subsection{Attacking, Threat, and Preparation}
When making any attack, you roll [dice] to hit your target. When you hit your target, you deal damage equal to the weapon’s damage.

If you roll maximum when trying to hit your target, you can increase the level of damage that you deal, called a critical attack.

If you have a melee weapon, you threaten the area around you at all times.

If you have a ranged weapon, you threaten the line-of-sight area between you and your target. If you have attacked a target with your ranged weapon, you threaten them until the start of your next turn.

If you have a melee weapon and you have prepared an attack on a target, you threaten the target as long as they are within your movement range, even if they are out of your sight.

If you have a ranged weapon and you have prepared an attack on a target, you threaten them at all times.
\subsection{Basic Maneuvers}
Basic maneuvers are things from other role-playing games that give you
advantages in situations. Sneaking up on an opponent before an attack,
disarming an opponent, and ganging up on enemies are examples of maneuvers. Any
character can perform any basic maneuver, granted that they meet the
requirements to perform it and the proper conditions are met.
\subsubsection{Take Cover}
Taking cover is usable at all times.

It decreases the severity of damage that you take from all ranged attacks. All wounding hits become glancing hits, and all critical hits become wounding hits.

If you take cover, you are no longer threatened by opponents with ranged attacks as long as they no longer have sight of you.

Advanced Take Cover Maneuvers: Vault
\subsubsection{Flank}
Flank is usable against people who are threatened from two opposing sides. Note: you can be one of those threatening sides. Flank is also usable against threatened opponents who are unaware of your presence regardless of threat.

Flank grants a +2 bonus to hit your target. This bonus cannot allow you to critical attack; you may critical attack normally.

Advanced Flank Maneuvers: Sneak Attack, Ambush
\subsection{Advanced Maneuvers}
\subsubsection{Sneak Attack}
Training Required: 1 Stealth Mark.

Sneak Attack improves the Flank maneuver.

When you use Flank, you gain a +2 bonus to your damage in addition to the +2 bonus to hit that you get from Flank.

Cost: 1 momentum.
\subsubsection{Ambush}
Training Required: 2 Stealth Marks, Sneak Attack

Ambush is usable against threatened opponents that are also unaware of your presence.

When you use Ambush, you gain a +3 bonus to hit. This extra hit can allow you to critical attack; apply the extra bonus to any combination of dice involved in the attack totaling to +3. You automatically deal maximum damage with your attack.

Cost: 2 momentum.
\subsubsection{Vault}
Training Required: 1 Running Mark

Vault improves the Take Cover maneuver.

When being fired upon by a ranged weapon, you may immediately gain the benefits of any piece of cover around you.

Cost: 2 momentum
\subsubsection{Survival Instinct}
Training Required: 1 Running Mark, Vault maneuver.

Whenever you would use Vault, you may immediately move your base movement speed before resolving the attack, taking normal movement speed penalties (and therefore making any checks) for any obstacles you wish to cross. Note that, after your move, you may still be both threatened and unaware of your opponent’s attack.
\section{Rewards}
\subsection{Equipment}
\subsection{Experience}
\section{New Tokyo Setting}
Although there are many different cities and runner groups in the world, New
Tokyo is a uniquely diverse city. Since it developed relatively quickly and was at the peak of its socio-economic development during the Privatization Era and the November Riots, New Tokyo became an example for the rest of the world.
\subsection{Organizations}
\subsubsection{New Tokyo Runners}
\subsubsection{City Protection Force}
The CPF is the main police force of New Tokyo.
\subsubsection{Pirandello Kreuger}
PK is the private security force of New Tokyo, which was recently awarded a
city-wide contract in assisting the CPF with security. Since they are a private company, they get around a lot of the laws that the CPF must follow, making them a dangerous ally. They were also the company that helped develop Project Icarus. As CPF tries to exterminate the runners, PK lends the CPF their Icarus forces, finally allowing them to strike back at other people.
\subsubsection{Project Icarus}
\subsubsection{Transit Authority Organization}
The TAO is the cross-continental organization designed to provide quick
transportation between cities and regions. Since New Tokyo is such a hub for business, the TAO has taken control of the largest thoroughfares through the city to facilitate inter-city travel more effectively. New Tokyo is a TAO hub for preparing trains for cross-region travel, and as such, all the TAO tunnels and trains have a lot of surveillance around them.
\subsection{Important People}
\subsubsection{Faith Connors}
\subsubsection{Drake (Deceased)}
\subsubsection{Jacknife (Deceased)}
\subsubsection{Tracy}
\subsubsection{Sparrow}
\subsubsection{Robert Hope (Deceased)}
\subsubsection{Kreeg}
\subsubsection{Lieutenant Miller (Deceased)}
\subsection{Districts}
\subsubsection{Pinnacle District}
Pinnacle is the downtown of New Tokyo, with many large office buildings and
malls all around. The Mahjoud Park is an important section of the Pinnacle
District, with the busiest shopping center adjacent and the associated subway
transit station underground. The district is located in the north of New Tokyo,
and even though the Perimeter Transit system is the supposed boundary of the
city, construction of new buildings on the border of the Pinnacle District is a
common sight.
\subsubsection{Artemis District}
The Artemis District is where the technology of New Tokyo is developed.
\subsubsection{Francisco District}
The Francisco District is a vacation resort in a city; adjacent to New Pinnacle
District and the Kosho River, it makes for a beautiful tourist destination.
With the view of the “historic” Brennon District just across the way of the
river, the most famous and well known spot is the Xianguo Resort, with craggy
rocks jutting next to the edge of the river.
\subsubsection{Brennon District}
The Brennon District was the first completed neighborhood in New Tokyo; although
businesses have moved into the older government buildings, the Brennon District
never really escaped the fate of becoming the “abandoned government district.”
Crime rates have been steadily increasing, but it’s still the least monitored
district in all of New Tokyo.
\subsubsection{New Pinnacle District}
The New Pinnacle District is essentially the club district; where Pinnacle is
commercial, New Pinnacle is more of the high-end seedy-underbelly of New Tokyo.
Its proximity to the Francisco tourist district makes it a place for tourists
to party with the locals.
\subsubsection{Government District}
The Government District is dwarfed by the huge Citadel; the buildings are still
among the largest and grandest in all of New Tokyo.
\subsubsection{South Government District}
This is a new government district, opened to be closer to the Chute. So many
administrative tasks happen around there, that it is important to have.
\subsubsection{Tokyo District}
The Tokyo District is a transplantation of some of Old Tokyo’s old famous
attractions. Although main and important services have been transferred to
other districts, Tokyo District still has the old Tokyo feeling that is missing
from the rest of the city.
\subsubsection{Kreuger District}
The Kreuger District is one of the most secure districts in New Tokyo;
surveillance is everywhere, and most of the district itself is walled off,
owned by Pirandello Kreuger. However, the small stretch of the Kreuger District
that is open to the public is one of the best places for people to do business;
Kreuger always has some sort of recorded evidence of the goings-on.
\subsection{Landmarks and Locations}
\subsubsection{Pinnacle Avenue}
The main thoroughfare through New Tokyo, Pinnacle Avenue goes straight from the
Pinnacle District to the heart of the city’s transportation systems. Pinnacle Avenue combines with Highway 45 in the very center of the city to create what has been nicknamed the Chute; a system of transportation that is extremely efficient at moving people. Pinnacle Avenue is also famous for the Golden Bridge, the expansive bridge in the heart of the Chute that crosses the Kosho River. At this point, Pinnacle Avenue crosses the river, moving from the north side to the south side of the river. Eventually, it meets up with Highway 45 in the south part of New Tokyo.
\subsubsection{The Chute}
The Chute is the main transportation point in New Tokyo. The TAO
inter-continental trains, along with all local automotive traffic and subway stations, all hook into the Chute. It is a hub of activity, with cars and trams all stopping in the Chute to pick up new passengers. There is also a tram system that runs from the Chute to the TAO airfield.
\subsubsection{Artemis Prime}
\subsubsection{Mahjoud Park}
The park is often crowded after hours, with people stopping to take a break on
their way home. As such, the park has heavy surveillance around it. The shopping center is often filled with private security guards, although the City Protection Force is often spotted taking shifts on the center’s grounds.
\subsubsection{Kosho River}
The Kosho River flows south into New Tokyo, entering the city under the east
Highway 45, and leaving the city in the Brennon District to the west. A decadent resort has been opened in the city along the riverside in the New Pinnacle district, the Xianguo Resort, catering to high-class businessmen stopping in New Tokyo. The government passed many laws upon the creation of city construction around the river to prevent pollution and marring of the river. Some of these laws include no boat launches; the river itself remains an uninhabited stretch of water, save for the CPF patrol boats.
\subsubsection{Xianguo Resort}
A resort very recently opened that is along the Kosho River. Although the
Xianguo Resort has no official river access, people climb down large rocks that the resort has placed close to the riverside, gazing over the historical Brennon District. The resort itself is in the Francisco District.
\subsubsection{Highway 45}
Highway 45 is one of the most important railways through Asia. It goes through
New Tokyo, through the Chute and the Junction, but most people consider New Tokyo the last important stop along the 45. The freeway through the city is large; onramps are skillfully woven into the city, and the infrastructure under the highway grants easy access to the runners; it’s a catacomb of twisting service passageways and support beams that allow the runners to lose people chasing them.
\subsubsection{The Junction}
The Junction is the southern intersection of the Perimeter Transit System and
Highway 45, in the south end of New Tokyo. Although it isn’t as large as the Chute, it’s an important nest for runners; its distant location from both the South Government District and the TAO means that, although there’s little true crime, the runners can get away with more here. A lot of information is transferred, dropped off, or picked up somewhere near the Junction. As Icarus units start to patrol the Junction, the runners have learned to stay away, but old habits die hard, especially when the Junction is still the easiest place to run.
\subsubsection{TAO Railyard}
This is where all the TAO train cars are stored.
\subsubsection{TAO Airfield}
This is where all the TAO planes are stored. All air traffic goes through this
airfield; although private chartering is not unheard of, it is uncommon these days. Most inter-city travel goes through the TAO.
\subsubsection{Perimeter Transit System}
The Perimeter Transit System is the official edge of New Tokyo. A quick and
efficient commuter-transit system runs along the outer edge of New Tokyo, with the three largest and most important spots being the stop in the Pinnacle District, the stop along Pinnacle Avenue near the Chute, and the stop at Highway 45 near the Junction. The Perimeter Transit System occasionally overlaps with the Transit Authority Organization, making use of the high-speed rails located in New Tokyo to shuttle people quickly, especially near the Chute and the Junction.
\subsubsection{The Citadel}
The most imposing building over the entire city, the Citadel is located right
next to the Pinnacle District. It is the CPF headquarters, with an extensive prison system running underground. A subway system links up to an island in the Pacific, where the most dangerous criminals are kept.
\subsubsection{Ashton Harbor}
When anybody thinks about anything boat-related, they’ll find it at Ashton
Harbor. There are two things about the harbor that are important. One is the docks, where commercial ships load up large amounts of goods. The large boats and loading machinery make getting around the harbor easy. The other part of the harbor that is important is the ship graveyard; ships that are being destroyed are near the docks. A lot of times, people want parts from the salvaged ships before they’re scuttled for good and turned into scrap metal. Security on the harbor is high; security on the scrapyard is low.
\subsubsection{Something Stadium}
\subsubsection{Indoor Zoo of Some Sort}
\subsubsection{November Memorial}
\subsubsection{The Abandoned Quarry}
The Abandoned Quarry is literally a runner’s playground. The biggest problem
with the Abandoned Quarry is that it is slowly become less abandoned; the city of New Tokyo is considering putting the quarry back into operation for construction materials for nearby cities. The easiest part of the quarry are the jobs; someone will often arrange transportation for the runners to get to the Quarry. The runners often just need to do their run, grab something from the Quarry (often some abandoned piece of technology), and get back. The Quarry is often used as a training of sorts for novice runners.
\section{Sample Adventures}
\subsection{The Icarus War: Origin}
\subsubsection{Prologue}
The evening news comes on, and a reporter stands at a grand statue, two large intertwined rings built in the middle of an open meeting square. She stands with a reporting crew; her head tilted slightly to the side, and begins speaking. “It has been just a single month since Robert Hope’s murder and the reveal of the November Riot memorial in the wake of his death. Petitions to rename the memorial the ‘Hope Memorial’ have been signed by over thirty thousand of the Pinnacle District residents, but so far, the government has not responded to the petition.”
“Most people remember the conspiracy surrounding Robert Hope’s murder, and the corrupt CPF officer that committed the crime. Kate Connors, one of Lieutenant Miller’s most respected officers, was first on the scene and was charged with Hope’s murder. When detained, she claimed that she was set up and that Hope was dead when she arrived on the scene. Kate’s sister, a wanted criminal, sabotaged the CPF vehicle transporting her to prison, and the two of them escaped. Kate and her sister struck again, killing Miller, before sabotaging a CPF helicopter above the Citadel.”
“Kate and her sister belong to a band of domestic terrorists; the Runners, a small group of people sabotaging the CPF and the order in the city of New Tokyo. They have been responsible for running drugs into the city, stealing confidential information from corporations and the government, and selling it to the highest bidder--”
A shadow passes briefly overhead, and someone yells. Birds chirp and caw as they scatter from the nearby rooftops. Looking up, the camera picks up a person jumping between buildings, sailing over the camera crew below. A loud shout is heard, “throw me the pack!” A bright yellow canister sails over the camera crew, and people take off. CPF sirens follow, and the action begins!
\subsubsection{Scene 1: The Life of a Runner}
Everyone takes part individually, and has a team partnership with one other person. The two people in scene are always a team; one holds the package, and one is the bait, and there is always a handoff. GMs should pick two people who are experienced or up to introducing the challenge. Each player should perform a few running checks, and a few fighting checks. Pass off the canister between two players, and have one person exit the scene, and have a new person enter. If examples for teamwork happen, try to encourage the players to work together.
When the canister arrives at its destination, players get together and investigate it. It was actually a package for the Runners; special pickup and transport to the safe house they’re currently at. Upon opening it, they find some technology tools that the runners have been waiting for, and some news stories and personal accounts about the Mirror’s Edge incident. “Red freedom, remember November!”
\subsubsection{Scene 2: Running the Core}
After the introductory run, the runners eventually reconvene and are asked to
run “The Core.” It is a new shopping center being opened in New Tokyo’s Pinnacle District. The biggest problem about the Core is that nobody knows anything about the layout. The runners will have to find out just how the Core is laid out by analyzing it, stealing documents from the construction company, looking at information, and by looking at advertising pictures and looking at it physically.
\subsubsection{Scene 3: Pirandello Kreuger}
\subsubsection{Scene 4: Icarus Conflict}
\subsubsection{Scene 5: Meeting With Drake}
\subsubsection{Scene 6: Icarus Takes Flight}
\subsubsection{Finale: Flight Delayed}
\subsubsection{Epilogue}
\subsection{Peace in Hell}
Although the government has turned the Runners into a terrorist organization,
some realize that the Runners are not all bad. \emph{Peace in Hell} is a sample
adventure that shows that not everyone in the government is out to destroy the Runners.
\subsubsection{Prologue}
The prologue should be a 15-20 minute simple mission having the runners escaping
from CPF forces, dropping off information somewhere. Easy, simple, and if you use this prologue, you can reference it later.
\subsubsection{Scene 1: A True Terrorist Attack}
Players start the game approaching a plaza on the rooftops of New Tokyo. A biological weapon was released in Pinnacle Square, and people have been quarantined. The runners are beginning their own investigation, and people in the other districts have begun to riot since they have not been told any information.
As the runners get closer, they see someone in runners’ clothes with a CPF emblem standing on the building the runners are running towards. If they approach her, she turns and faces them and asks them for help. If the runners turn tail, she chases and calls after them.
The CPF agent asks the rest of the runners for help. The biggest problem is, because of the government censorship, there is a general lack of information to the public about the terrorist threat. The CPF agent asks the runners to evacuate the city peacefully in order to avoid a repeat of the November Riots.
The runners can descend from the buildings and try to incite the crowd, or they can use CPF-provided megaphones to direct the crowds from the buildings above.
\subsubsection{Scene 2: Icarus Officer Stella}
Immediately after the resolution of the problem, the CPF officer climbs the building again and talks with the runners again. This time, she’s also wearing an Icarus emblem; she introduces herself as Icarus Officer Stella.
Stella proposes a deal with the runners. Icarus stops going after the runners, and they help the government find out who really attacked Pinnacle Square. The runners are in an ideal position to find more information, as they transported the information that the terrorists used, and the Stella gives the runners her word that the government will keep their end of the deal.
\subsubsection{Scene 3: Investigating the City}
The biggest problem that runners will have while investigating the city is that few people trust the runners, having been branded as terrorists by some and freedom fighters by others. The government protection for the runners doesn’t extend to NPCs the players find; if the government finds criminals that are working with the runners, then they may be arrested, or the runners may have to work out the situation.
Shipping records, money wire orders, etc. are the kinds of sensitive information that runners normally transfer. Looking for those sorts of things should be the way the players find the building. There are two main pieces of information that the runners need: the district that the building is in, and the cover company that was the front for the real terrorist organization. Also discoverable is the name of the terrorist organization, although it is difficult to find it.
\subsubsection{Scene 4a: District Hunting}
If the runners investigate the crime scene, or manage to go through their records (recently seized by CPF/Icarus in a recent raid), they can discover from which district the attack originated. The machine used to deploy the bio-weapon in the three locations was technologically advanced, which should lead most runners to Artemis. However, if the runners make good on going to Artemis to investigate further, they won’t find anything; they need the specific part of Artemis, although some other clues can lead them to the proper part of the Artemis district.
The machine itself is technologically advanced and bears markings of being manufactured in the Artemis district. However, on certain parts of the device, there are telltale manufacturing imprints from Sirta Industries, which has major manufacturing plants in the Government District and Sirta 45, a large factory located under a small stretch of Highway 45 (which is next to the Artemis District).
When at Sirta 45, players can sneak in and find records that they need, or they can try hacking their communications for more information. Direct sabotage, or even having the CPF arrest (and interrogate) managers at Sirta 45 are all valid options, although if the players get creative, run with it. Eventually, they should learn that the distribution device was from Artemis Prime, a strip of the Artemis District that is adjacent to Highway 45’s Sirta Foundation. However, because of the illegal activities, the device itself doesn’t have any markings implicating corporations involved in the attacks.
\subsubsection{Scene 4b: Company Hunting}
Trying to find any paper trails leads to dead ends; they always end up pointing to well-established corporation shards, and they are all different shards. If the players ask CPF about it, they report that corporate corruption points to specific “shards” that deal with less-than-legal activities, so that if the corps need to burn ties they can. However, if players ask Stella and CPF to push harder, they find that one corporate shard is different: Itelli Corporation’s Biolab Division. Most corporate corruption shards are biology based, both because of their delicate testing nature and their huge expenses make it easy to hide illegal activities. Itelli Corporation deals mainly in security construction and reinforcement, which makes a biolab division curious.
The more common thing that players may run into a CPF force that is trying to pursue a fellow runner. Lethal force has been authorized, and the runners are given a brief description which ends up being throwaway (make something up on the fly). CPF then goes on the move, trying to cut him off; the runners can take to the rooftops to try and pursue, or can join CPF and see if CPF cuts him off properly.
If things get really bad, Icarus is called in, and at some point, there’s a firefight between a small group of the terrorists, Icarus, and the players chasing the lone wolf across the firefight, possibly getting shot at. Players can opt to handle the lone wolf, try to help CPF and Icarus overwhelm the group of three terrorists, or simply get out of there. The terrorists have advanced weapons and are very dangerous; the players should get injured and wounded if they attack them directly or head-on; confusing, flanking, and distracting gives the CPF heavy forces, Icarus units, and other players an advantage to subduing the terrorists.
Catching the lone wolf is very different; because he is as agile as the players, they must work cooperatively to trap him somewhere that he can’t escape. When they do, the players can try to take him down and arrest him themselves, or they can just trap him and chat while waiting for CPF to bring the proper equipment to arrest him. During this chat, they can actually get the information they need; the runner calls himself Heron, and said that he’ll tell the players anything they want to know; he runs for the fun of it, and getting caught is one of the dangers of the job. Especially since he’s kept his hands clean (at least directly), he figures he may get a slap on the wrist or a little bit of prison time for being caught. When asked about his involvement with the terrorists, he responds, “They paid me to run, so I did.”
If there terrorists are apprehended, they all mention “the organization,” which is interpreted as a codename. It makes actually finding the organization very difficult, but makes talking to people about it much easier. “The organization,” plus a code word, normally means people assume you’re in.
\subsubsection{Scene 4c: Finding The Organization}
Finding the Organization is difficult, but not impossible. If the players have
found everything, like Sirta 45, Itelli Corporation’s Biolab division, and subdued the terrorists and/or Heron, then it should be easy to discover who the Organization is. Enough people within the Organization probably know of your investigation, and end up reporting to the CPF information for reduced sentences. Finding the warehouse should be easy.
\subsubsection{Scene 5: Running Artemis Prime}
\subsubsection{Scene 6: Stella Betrayed}
After running Artemis Prime and the Omega Warehouse, a terrible news report is released. Stella, a high ranking officer in a special CPF task force has been reported kidnapped by runners, who have been blamed for the bio-terrorism attack. All the evidence that pointed to the Omega Corporation has been edited to put the runners at blame, including photos and descriptions.
Not long afterwards, the runners are sent a mysterious package marked with a CPF emblem and an Icarus Star, placed at a secure drop site in New Tokyo. Upon opening it, the players find a letter from someone.
“There is little time left so I’ll keep this brief. Stella’s been kidnapped by Omega (probably sold out by someone in the CPF); higher ups must’ve decided that she was more useful as bait instead of an officer. Take a look at this memo; it’s internal only. They want her dead, and they’ve taken her family as collateral. The reason I’m sending you this; not everyone in the CPF or Icarus believes in the cause, not anymore. They said you killed the innocents in Mahjoud Park, but when they betrayed us to get to you, I stopped trusting them. I’m out. Good hunting.”
Enclosed in the package are all of Stella’s findings about the Runners and Omega, along with the internal memo that the CPF sent out. The memo reads as follows:
“For your eyes only: Stella’s secondary mission to destroy the runners has been compromised, since she has been kidnapped by Omega or the Runners. Should Stella be recovered, she should be arrested for conspiring against the government; we will be arresting her family as conspirators in the terrorist attack. Recommend having Stella killed in crossfire with Omega, Runners, and CPF forces. Advise.”
The players are also given blueprints, plans, and recommendations for the CPF/Icarus attack on an Omega outpost located in the Francisco district. Players can opt to rescue Stella early, or go after Omega at the same time the CPF goes in to “rescue” her.
\subsubsection{Finale: Attacking Icarus}
\subsubsection{Epilogue}
\section{Quick Start Guide}
So you’ve got a character in front of you, and you’ve gotten the low-down on how some of the systems work. Here’s a quick review of the history of the world.
You’re a runner. What does that mean? You transport things across the city, anonymously. In today’s world, everything’s monitored; your work allows secrets to remain that way while still being transported between people.
You’re a hero, sort of. “Champion of the people,” some underground news reports call the runners. “Terrorists and rebels” is what the government-controlled media says about you. How you act will shape how the runners are seen.
You’re outgunned and outmatched. Every runner has at minimum a small amount of training and a lot of on-the-job practice. Every police officer you find will have loads more conventional training than you. You will be outpunched, outgunned, and outmaneuvered in almost all textbook situations; make sure the situations aren’t textbook!
You know the city. Policemen know the city’s streets; you know the city’s alleys and rooftops. Things normal people don’t know exist are right around the corner. You can scramble up pipes and ladders faster than any policeman; you’ve crawled through your fair share of air ducts.
You’re wanted. The city knows you, sharing its secret side-alleys and hiding spots with you. The policemen know you too, and they’re out to get you. Those of you who have been running forever are familiar faces on that wanted board; those of you who have started running recently don’t even have warrants yet. Don’t ever let your guard down.
There are some important things you should know about New Tokyo:
The City Protection Force, or the CPF for short, or the “blues” for slang, they’ll be your enemies. Any chatter involving them; listen up, cause they’re probably coming for you.
The Icarus Project is a CPF unit designed to pursue runners. If you hear squaking of Icarus, it’s time to run (and run hard)!
When in doubt, running away is probably the right answer. Don’t be afraid to flee!
Because you’re runners, you can’t get a whole lot of equipment when it comes to weapons. Technology and defensive equipment will be your specialties; things like hand grenades and assault rifles will be extremely hard to find.
\subsection{How to Play}
You’ve got two dice-related stats; traits and skills. The five traits are Strength, Agility, Intelligence, Spirit, and Instinct; there are numerous skills that your character may have some ability with.
Whenever you’re asked to make an action roll, you will roll a Trait die and a Skill die, and using the higher die as your result. Your goal is to roll as high as possible, trying to beat a target number.
Faith Connors is trying to disarm a CPF policeman. Since she’s got martial artist training, she’s going to be relying on her instincts and her disarm practice. She’ll roll her Instinct die and her “Disarm” skill die, and uses the higher of the two as her action’s result.
\subsection{Momentum}
As a runner, you often build up speed or build on previous successes in order to progress. Momentum is this concept; you use it in game to help aid you. Here’s a brief summary of things momentum is used for.
Before: You can spend 1 momentum to get a +3 bonus to your roll, and negate the effects of a botch.
Before: You can spend 1 momentum to break a larger skill die into smaller skill dice. The maximum values of the dice must sum to be equal to the broken die. These count as one die!
After: You can spend 1 momentum to reroll all your dice. You keep the new result, even if it is worse. You cannot reroll botches!
After: You can spend 1 momentum to get a +1 bonus to your roll. You cannot add to botches!
On Use: Some maneuvers require you to spend momentum to use. Some abilities also require you to have a minimum amount of momentum before you can use them.
\subsection{Gaining Momentum}
Gaining momentum happens at a few key points during scenes. Below are the most common ways you gain them.
Opening: Whenever you start a scene, you gain two momentum immediately. Certain conditions will change this, and you may gain more or less momentum.
Max Roll: Whenever a die you roll shows the highest value, you gain a momentum.
Teamwork: A teammate can give you momentum when assisting you.
\subsection{Prep and Threat}
There are some basic rules to threatening opponents and being threatened, but first and foremost it makes sense. Here are some of the basic rules of threat, so that if you get into combat, you can better fight back.
Threat is one way; a character threatens another character.
A character can only threaten one character at a time.
If two characters threaten each other, it means they are aware of each others presence, essentially “cancelling” threat advantages.
A character with a melee weapon threatens all targets within melee range.
A character with a ranged weapon threatens all targets within line-of-sight, along with a target they have attacked in the last turn.
A character with a melee weapon and a prepared attack against a target threatens them if they remain with movement range of the first character.
\subsection{Actions and Movement}
You perform an action on your turn. Movement is relatively loose; you move your
character as much as you think is possible. If you wish to move an extra distance, you will probably have to roll some sort of movement skill in order to keep moving.
\subsection{Damage}
There are three types of damage: glancing hits, wounding hits, and critical hits.
Glancing hits represent scrapes, bruises, and damaged mental health. Skidding on the ground, and bullets whizzing by your head are examples of glancing hits.
Wounding hits represent pretty major injuries, but nothing your character can’t handle. Getting thrown onto the ground by an enemy, getting shot in the arm, or spraining your ankle jumping are examples of wounding hits.
Critical hits represent crippling or life-threatening injuries. Getting shot in the leg or the gut, falling off a building, or being beaten unconscious are examples of critical hits.
If a character has taken the maximum number of hits of any given level, then hits of that level do the next level damage of severity. This is called wrap-around damage.
Rule of severity: if a character takes a wounding hit, they also take a glancing hit. If a character takes a critical hit, they also take a wounding hit and a glancing hit. Wrap around damage does not overlap with the rule of severity; rule of severity takes precedence and does not trigger wrap-around damage.
If any of your damage tracks are maxed, you take penalties to your actions. If your glancing hits are full, then you take penalties to perform actions. If your wounding hits are full, then you take more penalties. If your critical hits are full, then you are near death and fall unconscious.
\section{Sample Character Sheet}
\subsection{Martin Wells}
\subsubsection{Stats}
Strength: d6
Agility: d10
Intelligence: d8
Spirit: d4
Instinct: d8
Reputation: 2
\subsubsection{Health}
5 Glancing Hits
3 Wounding Hits
2 Critical Hits
\subsubsection{Maneuvers}
Instinctive Run: Whenever you would fail an acrobatics roll, you may reroll it.
Sneak Attack: Whenever you attack an unaware target, you gain a +2 bonus to hit and a +2 bonus to damage.
Sprinter: When rolling the Sprint skill, you may roll two Sprint dice and take the highest of your attribute die and your two skill dice.
Sure Footed: Whenever you fail an acrobatics roll, you may immediately spend one momentum to reroll it with a +3 bonus.
\subsubsection{Skills}
Acrobatics: d8
Martial Arts: d4
Sprint: d6
Stealth: d4
\subsubsection{Special Powers}
City Traversal: Once per game, Martin can sprint without rolling for distance. The GM determines the maximum distance you can sprint.
Spirit of the City: Once per game, everyone within Martin’s average range of movement (including himself) can immediately disengage from combat and sprint, regardless of turn order.
\subsection{Charlie Sanders}
\subsubsection{Stats}
Strength: d4
Agility: d10
Intelligence: d12
Spirit: d4
Instinct: d6
Reputation: 0
\subsubsection{Health}
4 Glancing Hits
3 Wounding Hits
1 Critical Hits
\subsubsection{Maneuvers}
Instinctive Run: Whenever you would fail an acrobatics roll, you may reroll it.
Sneak Attack: Whenever you attack an unaware target, you gain a +2 bonus to hit and a +2 bonus to damage.
Sprinter: When rolling the Sprint skill, you may roll two Sprint dice and take the highest of your attribute die and your two skill dice.
Sure Footed: Whenever you fail an acrobatics roll, you may immediately spend one momentum to reroll it with a +3 bonus.
\subsubsection{Skills}
Hacking: d8
Acrobat: d8
Infiltration: d4
Rigging: d4
\subsubsection{Special Powers}
Electronic Sabotage: Once per encounter, Charlie can declare an electrical component to sabotage in advance. If Charlie does not declare sabotage in advance, she may use Electronic Sabotage as she encounters a device in game.
Communication Overmastery: Once per game, Charlie can block all enemy communications in the area, while tripling the range of friendly communications.
\end{document}